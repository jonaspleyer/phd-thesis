\pagebreak
\chapter{Further Applications}

Kapitel 4 beinhaltet Anwendungsbeispiele. Mein Ziel hier ist, dass jedes dieser Beispiele
(subsections) mit dem Hintergrundwissen von Kapitel 1-3 lesbar ist. Dabei möchte ich hier
vor Allem zeigen, wie cellular\_raza verwendet wird und wie man die entwickelten
mathematischen Methoden auf die jeweiligen Beispiele anwenden kann. Inwiefern jedes der
Beispiele sowohl numerische Simulation als auch eine theoretische Abhandlung bekommt,
muss ich dann noch sehen.

%---------------------------------------------------------------------------------------------------
\section{Bacterial Branching}
\begin{itemize}
    \item Discuss Coarse-Graining
    \item Pattern Formation
    \item Hendrik Meyer - MOdeling semiflexible polymer networks\\
    \item Tom Witten - Diffusion Limited Aggregation 1980, 80er Jahre
\end{itemize}

%---------------------------------------------------------------------------------------------------
\section{Pool Model}

%---------------------------------------------------------------------------------------------------
\section{Cell-Sorting}

%---------------------------------------------------------------------------------------------------
\section{Trichome Patterning}

%---------------------------------------------------------------------------------------------------
\section{Puzzle Cells}

%---------------------------------------------------------------------------------------------------
\section{Sender-Receiver}

%---------------------------------------------------------------------------------------------------
\section{Power law - Infectious Communicating Bacteria}

%---------------------------------------------------------------------------------------------------
\section{Pear Rust Fungus?}

%---------------------------------------------------------------------------------------------------
\section{Clustering of ATG19 in Selective Autophagy}
