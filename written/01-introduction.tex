Kapitel 1 soll eine Einführung in ABMs geben. Dabei möchte ich vor Allem das Mini-Review
von uns verwenden und auf Gemeinsamkeiten, Unterschiede, und ein wenig Geschichte
eingehen. Hier möchte ich schon einige Ideen aufbringen, die später wieder aufgegriffen
werden. Diese habe ich erstmal als eigene Subsections gelistet.

%---------------------------------------------------------------------------------------------------
\section{Scales in Physics and Biology}
\begin{itemize}
    \item smallest proteins:
        \begin{itemize}
            \item \textit{spoVM} \cite{Levin1993} (sporulation in Bacillus subtilis)
            \item \textit{TAL peptide} (smallest naturally ocurring, discovered in Drosophila melanogaster in 2007) \cite{Galindo2007}
        \end{itemize}
    \item largest scales: population dynamics across continents (earth $\approx 40'000\text{km}$ circumference)
\end{itemize}

%---------------------------------------------------------------------------------------------------
\section{Cellular Building Blocks}
\begin{itemize}
    \item complex organisms from collection of cells
    \item self-organization/pattern formation
    \item single-cell vs bulk
\end{itemize}

%---------------------------------------------------------------------------------------------------
\section{Comparison of Simulation Frameworks}
\begin{itemize}
    \item use mainly mini-review for this subsection \cite{Pleyer2023}
    \item many frameworks
    \item some similar approaches
    \item purpose-built solutions
    \item missing flexibility in model design or ability to apply to various systems
    \item large number of parameters which need to be supplied
\end{itemize}

%---------------------------------------------------------------------------------------------------
\section{Mathematical Treatment}
\begin{itemize}
    \item no "unfiying theory"
    \item with theoretical framework, we would be able to systematically discuss any of the following:
        \begin{itemize}
            \item coarse-graining
            \item uncertainty analysis
            \item dimensionality reduction
        \end{itemize}
\end{itemize}

%---------------------------------------------------------------------------------------------------
\section{Are we flexible yet?}
\begin{itemize}
    \item flexibility matters (why)
    \item show some examples: rust pear fungus (differing cell types interacting), budding yeast
        elongated bacteria (non-trivial geometry)
    \item we are not there yet
\end{itemize}

%---------------------------------------------------------------------------------------------------
\section{Are we calibrated yet?}
\begin{itemize}
    \item many \acp{abm} do not do proper parameter estimation
    \item what are we actually modeling here?
\end{itemize}

%---------------------------------------------------------------------------------------------------
\section{The mesoscopic approach}
\begin{itemize}
    \item physicists: natural tendency to go big (many particles, large scale), or go small
        (fine-grained look at things, sub-microscopig)
    \item no real natural tendency to go mesoscopic
    \item we operate on the scale of 1-100 cells making it sometimes feasible to employ statistics
        and sometimes feasible to account for natural cell-cell variability
    \item depends on specific cases all the time
\end{itemize}
