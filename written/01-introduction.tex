Kapitel 1 soll eine Einführung in ABMs geben. Dabei möchte ich vor Allem das Mini-Review
von uns verwenden und auf Gemeinsamkeiten, Unterschiede, und ein wenig Geschichte
eingehen. Hier möchte ich schon einige Ideen aufbringen, die später wieder aufgegriffen
werden. Diese habe ich erstmal als eigene Subsections gelistet.

%---------------------------------------------------------------------------------------------------
\section{Scales in Physics and Biology}
\begin{itemize}
    \item smallest proteins:
        \begin{itemize}
            \item \textit{spoVM} \cite{Levin1993} (sporulation in Bacillus subtilis)
            \item \textit{TAL peptide} (smallest naturally ocurring, discovered in
                Drosophila melanogaster in 2007) \cite{Galindo2007}
        \end{itemize}
    \item largest scales: population dynamics across continents (earth $\approx
        40'000\text{km}$ circumference)
\end{itemize}

%---------------------------------------------------------------------------------------------------
\section{Cellular Building Blocks}
\begin{itemize}
    \item complex organisms from collection of cells
    \item self-organization/pattern formation
    \item single-cell vs bulk
\end{itemize}

%---------------------------------------------------------------------------------------------------
\section{Comparison of Simulation Frameworks}
\begin{itemize}
    \item use mainly mini-review for this subsection \cite{Pleyer2023}
    \item many frameworks
    \item some similar approaches
    \item purpose-built solutions
    \item missing flexibility in model design or ability to apply to various systems
    \item large number of parameters which need to be supplied
\end{itemize}

%---------------------------------------------------------------------------------------------------
\section{Mathematical Treatment}
\begin{itemize}
    \item no "unfiying theory"
    \item with theoretical framework, we would be able to systematically discuss any of
        the following:
        \begin{itemize}
            \item coarse-graining
            \item uncertainty analysis
            \item dimensionality reduction
        \end{itemize}
\end{itemize}

%---------------------------------------------------------------------------------------------------
\section{Are we flexible yet?}
\begin{itemize}
    \item flexibility matters (why)
    \item show some examples: rust pear fungus (differing cell types interacting), budding yeast
        elongated bacteria (non-trivial geometry)
    \item we are not there yet
\end{itemize}

%---------------------------------------------------------------------------------------------------
\section{Are we calibrated yet?}
\begin{itemize}
    \item many \acp{abm} do not do proper parameter estimation
    \item what are we actually modeling here?
\end{itemize}

%---------------------------------------------------------------------------------------------------
\section{The mesoscopic approach}
\begin{itemize}
    \item physicists: natural tendency to go big (many particles, large scale), or go small
        (fine-grained look at things, sub-microscopig)
    \item no real natural tendency to go mesoscopic
    \item we operate on the scale of 1-100 cells making it sometimes feasible to employ statistics
        and sometimes feasible to account for natural cell-cell variability
    \item depends on specific cases all the time
\end{itemize}

\section{Introduction}
\label{sec:self-organization}
% Self-organization~\cite{camazineSelfOrganizationBiologicalSystems2020} is an energy
% consuming dynamic process in which the collective behavior of individual agents
% exhibits emergent phenomena by forming spontaneous order through interaction between
% the agents, i.e., exchange of information, without the need of an external controlling agent.
% The emerging patterns cannot directly be inferred by the properties of the individual
% agents and are typically robust with respect to perturbations.
% Self-organization needs to be distinguished from self-assembly, which \nt{results from
% minimization of the free energy of a thermodynamic}
% system~\cite{johnAlternativeMechanismsStructuring2005,dambricourtmalasseSelfOrganizationNewParadigm2022},
% e.g., the self-assembly of a lipid bilayer or the spontaneous folding of
% proteins~\cite{kuhlmanAdvancesProteinStructure2019,albertsLipidBilayer2002,wedlich-soldnerSelforganizationFundamentCell2018}.
% \nt{In contrast,} self-organization happens in non-equilibrium conditions and are
% conceptually far more difficult to comprehend than equilibrium phenomena like
% self-assembly
% ~\cite{hakenScienceStructureSynergetics1984,ebelingPhysicsSelfOrganizationEvolution2011}.
% Examples for self-organization in
% biology~\cite{karsentiSelforganizationCellBiology2008} are social structures of
% insects~\cite{bonabeauSelforganizationSocialInsects1997}, developmental pattern
% formation~\cite{meinhardtModelsBiologicalPattern2008} and
% morphogenesis~\cite{collinetProgrammedSelforganizedFlow2021}.
% Self-organizing systems share common
% principles~\cite{bonabeauSwarmIntelligenceNatural1999}: the process of organization
% requires non-linear underlying dynamics, agents need to be able to communicate and
% interact, and a supply of energy to the system needs to be guaranteed
% \cite{karsentiSelforganizationCellBiology2008}.
% Interactions between cells can occur in a multitude of different ways and on different
% length scales: (short range) mechanical forces~\cite{viningMechanicalForcesDirect2017}
% and cellular junctions~\cite{bloemendalCelltocellCommunicationPlants2013}, (medium
% range) diffusing chemicals~\cite{duesterRetinoicAcidSynthesis2008}, and (long range)
% hormones~\cite{greenwoodGrowthHormoneSecretion1966}.
% Besides the exchange of information it is necessary that the cells (agents) process the
% signals and respond in a sufficiently non-linear manner, including feedback loops
% ~\cite{mitrophanovPositiveFeedbackCellular2008,deriteiFeedbackLoopConditionally2019},
% which can also provide robustness of the self-organizing
% structure~\cite{gerardEffectPositiveFeedback2012}.
% Self-organization expresses itself in the cellular reactions such as
% proliferation/apoptosis, migration, differentiation or in general change of behavior
% which are either modulated or activated as a response to signals transmitted by other
% cells~\cite{wolpertPrinciplesDevelopment2015}.
% In the context of cellular biology, self-organization is often synonymous with modeling
% spatial
% formation~\cite{deglincertiChapterSixSelfOrganization2016,karsentiSelforganizationCellBiology2008}.
% TODO talk about phenomena that we want to observe
Interacting cellular biological systems, such as bacterial communities, tissues,
organoids, exhibit a plethora of phenomena, which are often not easy to understand intuitively.
To explore and analyse cellular systems mathematical equations and/or computer
simulations can be powerful tools.
Because the fundamental building blocks or units of biological systems are cells,
agent-based models with cells as the individual agents are natural simulation tools to
study such systems.
Pattern formation in cellular systems requires interactions between cells, the exchange
of information, and, in case of self-organisation, that cells respond in a sufficiently
non-linear manner, including feedback loops
~\cite{Mitrophanov2008,Deritei2019}.
Exchange of information can occur in a multitude of different ways and on different
length scales: (short range) mechanical forces~\cite{viningMechanicalForcesDirect2017}
and cellular junctions~\cite{Bloemendal2013}, (medium range)
diffusing chemicals~\cite{Duester2008}, and (long range)
hormones~\cite{GREENWOOD1966}.
Due to the individual cell based perspective agent-based models make it easy to implement
these interactions and also the signal processing and response of the cells.
In this mini-review we consider agent-based software frameworks with individual cells as
agents, which are actively maintained and developed, provide documentation beyond a
minimum, and provide a model development workflow \footnote{An extended list of other
\acp{abm} can be found in the appendix.}.\\
Traditionally, pattern formation in biology is studied using \acp{pde} that model
continuous distribution of
cells~\cite{Kolmogorov1937,Turing1952,Koch1994,Gierer1972}.
There is a wealth of literature on how to solve coupled non-linear \acp{pde}, estimate
the parameters from data, and how to explore their behavior, e.g., sensitivity analysis,
bifurcation analysis, see, e.g.,
\cite{Browder1998,Muller2004,Stavroulakis2004,Baker2008,Saltelli2008,Kielhfer2012}.
\acp{pde} are powerful tools, however, in a \ac{pde} cells have no spatial extension,
ignoring the underlying cellular spatial structure.
Although it is possible in a \ac{pde} to distinguish between the inside of the cells and
their micro-environment, there is no unique or canonical way to handle, e.g., cell
proliferation, differentiation, internal cell structure and other properties of the
cells, which may be relevant for the questions at hand.
% BITTE FORMULIERE ES SO, DASS CA ZUERST GENANNT WERDEN (WAS WAR DENN DIE MOTIVATION?
% BIOLOGISCHE SYSTEME?) DANN ERKLÄRE WAS CA SIND.
% DANN KOMME AUF ABMS ZU SPRECHEN UND ZITIERE EIN FRÜHES PAPER UND NICHT NUR EIN PAPER VON 2019}\\
This problem can be solved by \ac{ca}~\cite{VonNeumann1966}
which consist of a regular grid with a finite number of states at each grid point and
rules, determining how to update them accordingly.
A further development of \ac{ca} introduced
\acp{abm}~\cite{Schelling1971,Reynolds1987,Glen2019}
where the modeling approach is to handle cells as individual agents with rules,
potentially with an internal structure and/or moving in space.
By using coarse-graining or homogenization techniques one could derive a system of
coupled \acp{pde} from an \ac{abm}; there is, however, no unique way to go from a
\ac{pde} to an \ac{abm} \cite{Nandakumaran2007}.

\section{Agent-Based Models}
\label{sec:abms}\noindent
%
% DEFINITIONS (C)ABM(F) AND CA + EXAMPLES
An \ac{abm} is a collection of autonomous agents with a predefined set of rules, which
depend on the existing state of the agent and external
factors~\cite{Bodine2020,Metzcar2019,Tomlin2007,Nagarajan2022}.
The rules can be discrete following logical if-else statements, continuous, i.e.,
\acp{ode} for intra-cellular reactions or a combination of both.
Also, graphs, neural networks and other intricate algorithms can be implemented~\cite{Haynes1996}.
Nevertheless, one usually strives to employ the most simple set of rules sufficient to
accurately describe the complexity of the desired system.
Compared to macroscopic \ac{pde} models, \acp{abm} are considered microscopic modeling,
since they deal with agents directly and are thus more common in a bottom-up
approach~\cite{Bonabeau2002}.
\acp{abm} should not be seen as a technologically distinct toolset but rather as a
mindset for researchers by modeling complex systems from the perspective of individual constituents.
\newline
%
Historically, precursors to \acp{abm} were \acl{ca}, which were developed
by~\cite{VonNeumann1966}.
They reached widespread recognition even in the general public with the introduction of
Conway's 'Game of Life'~\cite{Games1970,Berlekamp2001}.
Not long after, the first \acp{abm} were being envisioned to study a biological
system~\cite{Schelling1971}.
Up until the break of the century, \acp{abm} were used in many fields of research such as
modeling human crowd stampedes~\cite{Helbing2000}, bird
flocks~\cite{Reynolds1987} or the prediction of financial
markets~\cite{Kephart2000}.
With the rapidly growing accessibility and power of modern computer hardware, the
popularity of \acp{abm} kept on increasing, where tools such as
NanoHUB~\cite{Klimeck2008} or the
\ac{sbml}~\cite{Keating2020} further helped to share computational
models between researchers.
In order to study complex phenomena such as pattern formation \acp{abm} must be able to
capture cell-cell communication and cellular response
mechanisms~\cite{Bajpai2021,Nakamasu2009,Schnakenberg1979,Gierer1972,Wolpert2015}.
In the next section we will compare the available \ac{abm} frameworks and discuss how
they cover different cellular properties.
%
\subsection{Comparison of ABMs}\noindent
\label{subsec:abms-comparison-of-abms}\noindent
The effort of writing efficient solving algorithms and data structures in a usable
fashion is considerable.
Therefore, \acp{abmf} have emerged that define a certain workflow and implement a set of
features, so that users of the frameworks can focus on their research question instead of
having to spend a significant amount of time for design and implementation.
\newline
%
% WE PRESENT THE MODELS
The majority of \acp{cabmf} evolved as generalizations of solutions to specific problems.
% Biocellion~\cite{kangBiocellionAcceleratingComputer2014} simulates whole living-systems
% and is applied by companies in consumer research modeling
% cancer~\cite{aguilarGeneralizableDatadrivenMulticellular2020}.
BSim~\cite{Gorochowski2012} was specifically designed to model
bacterial populations and has been used to study gene regulatory
control~\cite{Martinelli2022} and bacterial biofilms~\cite{Jin2020}.
Chaste\footnote{We only consider here the cell-based part of the chaste software
environment.} was designed as a \textbf{C}ancer, \textbf{H}eart and \textbf{S}oft
\textbf{T}issue \textbf{E}nvironment~\cite{Cooper2020} and has been used
in studying growth of epithelial monolayers~\cite{Dunn2012}.
CompuCell3D~\cite{Swat2012} originated from
CompuCell~\cite{Izaguirre2004}, which was one of the first
frameworks created and originally used to model only simple \ac{rd} systems but was since
extended considerably to cover a wider range of topics such as
angiogenesis~\cite{Nivlouei2021},
cancer~\cite{Asadullah2021} and tissue engineering~\cite{Moldovan2021}.
EPISIM was used to understand how varying proportions of T-cells emerge in different
vertebrate taxa~\cite{Aghaallaei2021}.
% The authors of LBIBCell~\cite{tanakaLBIBCellCellbasedSimulation2015} aimed to
% investigate \ac{2d} morphogenetic cellular systems such as apical surface dynamics of
% epithelia~\cite{farhadifarInfluenceCellMechanics2007}.
% MecaGen~\cite{delileCellbasedComputationalModel2017} provided interesting case-studies
% about \ac{tp} formations in embryonic stages of zebrafish tissue formation.
Morpheus~\cite{Starruss2014} was applied to self-organization
in neural stem cell divisions in adult
zebrafish~\cite{Mulberry2020} and polarization of the
multiciliated planarian epidermis~\cite{Vu2019}.
MultiCellSim~\cite{Dang2020} resulted from the in-depth analysis
of cell-cell communication and was since applied to Immuno-Oncology~\cite{Karolak2021}.
PhysiCell~\cite{Ghaffarizadeh2018} is mainly used modeling cancer and
tumor dynamics~\cite{PoncedeLeon2022,Goncalves2021}.
TiSim/CellSys~\cite{Hoehme2010b} was applied to liver
regeneration processes~\cite{Hoehme2010c}.
VirtualLeaf~\cite{Merks2011} was specifically designed for modeling plants and emphasizes
intercellular connections and details of the mechanical properties of the cell wall.
%
% TALK ABOUT THE TABLE
Table~\ref{tab:abms-compare-abm-frameworks} displays general characteristics of these
modeling frameworks.
\begin{table}
    \centering
    \begin{tabular}{@{}lcccc@{}}
        \toprule
        Framework       &\makecell{Spatial Representation\\\& Dimension} &Intracellular
        &Extracellular &\makecell{Cell-Cell\\Forces}\\
        \cmidrule{1-5}
        % BioCellion      &\makecell{off-lattice\\\acs{2d} + \acs{3d}} &secretion \&
        % uptake &\makecell{\ac{rd} \acp{pde} with\\adaptive mesh resizing}
        % &\makecell{ellipsoidal/cylindrical\\ cell potentials}\\
        BSim            &\makecell{off-lattice,\\Arbitrary Meshes\\\acs{3d}}
        &\makecell{\acsp{ode}} &\makecell{\acsp{pde},\\Molecule-Agents}
        &\makecell{Micro-Scale Meshing and\\Collision Detection}\\
        Chaste          &\makecell{\acs{cpm}, off-lattice,\\\ac{ca},
        Vertex-Model\\\acs{2d} + \acs{3d}} &\acp{ode}, \ac{sbml} &\acs{rd} \acp{pde},
        \ac{sbml} &custom force laws\\
        CompuCell3D     &\makecell{\acs{cpm} on regular lattice\\\acs{2d} + \acs{3d}}
        &\makecell{\acp{ode}, \ac{sbml},\\\acsp{pbpk}} &\makecell{\ac{rd} \acp{pde},
        \acs{sbml},\\\acsp{pbpk}} &\makecell{force terms via\\\acs{cpm} hamiltonian}\\
        EPISIM          &\makecell{off-lattice, hexagonal\\\acs{2d} + \acs{3d}}
        &\makecell{\acsp{ode}, \acs{sbml}} &\makecell{\ac{sbml}} &spherical cell potentials\\
        % LBIBCell        &\makecell{off-lattice motion +\\\acs{cpm} on regular
        % lattice\\\acs{2d}} &\makecell{\ac{rd} \acp{pde}} &\makecell{\ac{rd} \acp{pde}}
        % &\makecell{Immersed Boundary\\method}\\
        % MecaGen         &\makecell{off-lattice\\\acs{3d}} &\acp{ode} &\ac{rd} \acp{pde}
        % &\makecell{Discrete Element method,\\ellipsoidal cell potentials}\\
        Morpheus        &\makecell{\acs{cpm} on regular lattice\\\acs{2d} + \acs{3d}}
        &\acsp{ode}, \acs{sbml} &\makecell{\ac{rd} \acsp{pde}, \ac{ca}
        lattice\\\acsp{ode}, finite state\\gradient-based} &\makecell{force terms
        via\\\acs{cpm} hamiltonian}\\
        MultiCellSim    &\makecell{\acs{ca} +\\Brownian motion} &secretion \& uptake
        &\acs{rd} \acsp{pde} &-\\
        PhysiCell       &\makecell{off-lattice\\\acs{2d} + \acs{3d}}
        &\makecell{\acs{sbml}, Boolean Networks,\\\acl{dfba}} &\makecell{BioFVM
        Reaction\\Kinetics} &Spheres with Potential\\
        TiSim/CellSys   &\makecell{off-lattice\\\acs{2d} + \acs{3d}} &\acsp{ode},
        \acs{sbml} &\makecell{diffusion +\\advection} &\makecell{frictional, elastic
        and\\stochastic force terms}\\
        VirtualLeaf     &\makecell{vertex model \acs{2d}} &\acsp{ode} &- &polygonal
        finite elements\\
        \bottomrule
    \end{tabular}
    \caption{
        Comparison of \acp{cabmf} in alphabetical order with respect to implementations
        of spatial representation, dimension, intra- and extracellular processes and
        cell-cell forces.
        For additional modeling tools (not necessarily \acp{abm}) see~\ref{appendix-a}.
    }
    \label{tab:abms-compare-abm-frameworks}
\end{table}

%
% SPATIAL REPRESENTATION
% - TRANSPORT
% - MIGRATION
% - ADHESION
% - CHEMOTAXIS
\subsubsection*{Spatial Representation}
\label{subsec:abms-spatial-representation}\noindent
A key distinction between \acp{abm} is given by the difference of the spatial
representation of cells and chemicals.
\acp{abm} can be separated into lattice-based and lattice-free, the former meaning that
cells can only migrate between predefined lattice nodes, while the later permits free
movement of cells in a given domain.
Frameworks such as Chaste, PhysiCell, TiSim/CellSys and VirtualLeaf utilize off-lattice motion.
Chaste\footnote{Cell-based Chaste supports off-lattice as well as on-lattice
representations.}, CompuCell3D and Morpheus utilize lattice-based methods for cell-migration.
This also means that no particular cellular shape is modeled explicitly, but rather cells
follow rules (often potentials) to determine their respective quantity on lattice points.
The disadvantage of the lattice-based approach is that it is limited in the spatial
resolution, but in turn as an advantage it can yield considerable performance improvements.
Off-lattice models often take a cell centre~\cite{Drasdo2007}
approach meaning, a cell is defined by a single location vector and a shape (such as
sphere, ellipsoid or cylinder) that governs interactions.
BSim additionally has the ability to represent microbes as meshed objects thus offering a
much higher resolution at micro-scale although at increased computational cost.
Another less common modeling choice is to use a vertex
model~\cite{Nagai2001,Smith2006} that represents
each cell by a polygon, determined by a number of vertices, which can be subject to
external forces, pressure, friction, adhesion, chemotaxis and other external and internal
contributing factors.
Lattice-bound models can utilize different discretizations such as regular Cartesian
meshes, hexagonal or triangulated ones.
Most of the presented frameworks in Table~\ref{tab:abms-compare-abm-frameworks} can be
used to simulate \ac{2d} as well as \ac{3d} scenarios.
%
%
The \acl{cpm}, also known as \ac{ggh}
model~\cite{Graner1992,Savill1997},
is a common choice for many frameworks.
Typically, in a Cellular Potts Model a Hamiltonian is formulated which describes the
phenomenological “energy” of a given configuration of the system on a Euclidean lattice.
Subsequently, the systems is evolved by minimizing the energy.
LBIBCell modifies the classical \ac{cpm} approach by representing cells as evolving
polygons with the immersed boundary method and thus obtains off-lattice cellular
representations~\cite{Drasdo2007,Peskin2002}.
%
% INTRA-/EXTRACELLULAR REACTIONS
\subsubsection*{External Microenvironment}
\label{subsec:abms-intra-extra-cellular-reactions-secretion}\noindent
Transport processes of chemicals typically involve numerically solving (convection-)
diffusion
equations~\cite{Alberts2002,Chandrasekhar1943}
with cell to extracellular matrix interaction nodes at the positions of the cellular
agents on a (often euclidean) mesh.
One exception is presented by VirtualLeaf where intracellular compartments are connected
via membranes to adjacent cells and model transport through membrane-potentials~\cite{Merks2011}.
Many \acp{abm} utilize \acp{pde} to model intracellular or extracellular transport
processes such as convection and diffusion and allow for custom forms of reactions either
via well-defined user-interfaces like
Morpheus~\cite{Starruss2014} or direct implementation into
the source code.

% Other models such as Biocellion~\cite{kangBiocellionAcceleratingComputer2014} and
% MecaGen~\cite{delileCellbasedComputationalModel2017} provide predefined formats for
% specifying reaction \acp{ode}.
% This limits possibilities but is a middle ground between writing source code and
% interacting only via a graphical user interface.
%
%
% CELL-SPECIFIC PROCESSES
\subsubsection*{Cellular Processes}
\label{subsec:abms-cellular-processes}\noindent
In an agent-based approach the processes occuring inside a cell can naturally be
described by giving the agents the required set of functions.
%
% - CYCLES
%   - PROLIFERATION
%   - DEATH
Each framework mentioned in Table~\ref{tab:abms-compare-abm-frameworks} implements
proliferation and cell-death mechanisms as key components.
However, predefined and detailed cell-cycle routines such as utilized in
PhysiCell~\cite{Ghaffarizadeh2018} are less common, but are important
to consider if, e.g., external factors such as growth hormones affect the
cell-cycle~\cite{Kassem1993}.
In addition, internal chemicals may be released upon cell death.
%
% - DIFFERENTIATION
In order to model developmental processes such as embryogenesis, the framework needs to
support cell-differentiation with dynamic modifications of the phenotype.
%
% - POLARITY
Cell polarity can play an important role in many phenomena such as in ciliary rootlets in
planarian epidermis~\cite{Vu2019}.
Many frameworks like CompuCell3D, Chaste, Morpheus, VirtualLeaf support this feature.
%
% - GEOMETRY/VOLUME
The geometry of the cell includes its spatial representation together with mechanical
features such as adhesion and repulsion.
PhysiCell utilize spheroid/ellipsoid cellular geometries, meaning each cell is
represented by a sphere or ellipsoid and a corresponding potential.
% Biocellion is also able to map other shapes such as cylinders to individual cells while
% Chaste and LBIBCell resolve cell-agents with multiple polygons subject to forces and
% internal pressure.
Further, adhesion plays an important role in cell-cell interactions and communication.
Lattice-free frameworks often model it by choosing a particular form of interaction potential.
One sophisticated example is the experimental \ac{jkr}
potential~\cite{Johnson1971}, which was derived from the Hertz
contact model~\cite{Hertz1882}.
It also models cell separation and is implemented by CellSys.
Other frameworks that implement a \ac{cpm} treat adhesion via interaction terms in its
Hamiltonian Formulation~\cite{Maree2007}.
In the context of vertex models, force potentials can also be utilized although the
implementation is often more complex.
% - MECHANICS
%
All of the above \acp{abm} are able to model stochastic cell migration, excluding
VirtualLeaf since almost all plant cells are non-motile.
Collectively arising forces and friction which can play an important role in early
embryonic development~\cite{Smutny2017} may be harder to simulate
if the geometry of the cells is solely implemented as spheroid/ellipsoid.
For frameworks such as PhysiCell and TiSim/CellSys who additionally do not support
polarity, modeling of force-mitigated spatial effects is difficult.
Chemotaxis is a key concept in cell-sorting~\cite{Vasiev1999} and
can be implemented by any framework that supports migration and can calculate reactant gradients.
% - INTRACELLULAR REACTIONS
All of the presented frameworks can capture intracellular reactions by using ODEs
ignoring the internal spatial structure of the cells; different reaction compartments can
be easily introduced by coupling of ODEs.
Some (e.g., Chaste, EPISIM) can also handle intracellular stochastic reactions, using the
Gillespie algorithm \cite{Gillespie2007}.
%
%
% TECHNICAL DETAILS
\subsection{Implementational Details}
\label{subsec:abms-technical-details}\noindent
\subsubsection*{Development, Standards and Features}
\label{subsubsec:abms-development}\noindent
Development and design of efficient algorithms and their implementation require knowledge
in software engineering and in writing maintainable code, as these frameworks are usually
developed by teams rather than by individuals and consist of many thousands of lines of code.
The Chaste framework was one of the first projects to follow agile coding principles and
other best-practice workflows such as rigorous unit-testing~\cite{Pitt-francis2008}.
%
All presented \acp{cabmf} are written in C++ which together with the C and Fortran
language have historically served as the de-facto languages for high-performance software
development.
%
In addition to \acp{cabmf}, researchers have over the last two decades developed
internationally recognized formats to seamlessly share model details (e.g. SMBL).
This is utilized in Chaste, CompuCell3D, Morpheus and PhysiCell\footnote{Via an addon
libroadrunner and only for intracellular reactions.} and allows for rapid model
development, implementation and comparison to classical \ac{ode} and \ac{pde} solvers.
CompuCell3D is also able to model \acp{pbpk}.
%
Additionally, many frameworks come with dedicated (sometimes \acp{gui}) tools for
configuration, analysis, batch-processing, visualization and other workflow-aiding
features which are valuable additions.
%
In this regard, EPISIM is special as it utilizes the popular COPASI~\cite{Hoops2006} and
Mason~\cite{Luke2005} software and plugins for the eclipse code
editor~\cite{Burnette2005} to build the application.

\section{Studying Pattern Formation with Agent-Based Models}
\label{sec:analysis}
%
%
\subsubsection*{Applications}
\label{subsubsec:abms-comparison-to-pdes}\noindent
Pattern formation in cellular biological systems can occur via self-assembly or
self-organization and \acp{abm} have been applied to investigate both aspects.
%
% CHASTE
Chaste was used to study cell migration in the crypt~\cite{Dunn2013}.
% LBIBCell
% 1. This paves the way for the development of data-driven 3D simulation frameworks that
% will be invaluable in the simulation of epithelial dynamics in development and disease.
% 2. BMP-receptor interaction meets the conditions for a Schnakenberg-type Turing
% pattern. We propose that receptor-ligand-based mechanisms serve as a molecular basis
% for the emergence of Turing patterns in many developing tissues.
% LBIBCell was applied in epithelial cell organization~\cite{iber3DOrganisationCells2022}
% and modeling BMP-receptor interactions~\cite{baduguDigitPatterningLimb2012}.
% CompuCell3D:
% 1. Physical forces among cells and between cell and substrate, along with mobility of
% individual cells, affect the self-organization process of leader cell formation during
% collective cell migration.
% 2. Work on polarization in migrating cells
Furthermore, CompuCell3D provided examples for self-organization in work on
polarization~\cite{Thomas2022} and studies of physical forces~\cite{Pan2021} in migrating cells.
% Morpheus
% 1. Pattern formation in zebrafish telencephalon
% 2. growth of the Drosophila wing via cell recruitment
% PhysiCell:
% 1. Pattern formation in tumour spheroids
Morpheus was used to describe pattern formations in the telencephalon of adult
zebrafish~\cite{Lupperger2020} and was also used to study growth of the Drosophila wing
via cell recruitment~\cite{Munoz-nava2020}.
PhysiCell recently provided insights to formation of patterns in tumour
spheroids~\cite{Goncalves2021}.
% VirtualLeaf:
% 1. Secondary vein patterning in dicot leaves
Pattern formation in dicot leaves was modeled using VirtualLeaf~\cite{Holloway2021}.
% Biocellion
% Biocellion followed a data-driven approach to model adenocarcinoma
% cancer~\cite{aguilarGeneralizableDatadrivenMulticellular2020}.
%
\acp{abm} allow researchers to examine complicated models which would otherwise be hard
to study and interpret with classical \acp{pde}.\\
Figure~\ref{fig:abms-comparison-physicell-pde} shows results of a multi-scale model using
PhysiCell~\cite{Ghaffarizadeh2018}.
We can observe that the pattern changes as the number of patterning cells (type I) increases.
This simple example shows, how to readily formulate and explore models in an \ac{abm}
mindset - by increasing the cell number in this case.
Constructing a corresponding \ac{pde} model is much harder and not uniquely defined.
%
%
\begin{figure}[h!]
    \centering
    \includegraphics[width=0.9\textwidth]{figures/abm-review/turing-pattern.png}
    \caption[Comparison of \ac{pde} and \ac{abm} simulation results]{
        We implemented a \ac{rd} system (see also \nameref{appendix}) in an \ac{abm} to
        showcase results.
        The simulation contains two distinct cell types, which are both motile and
        initially randomly distributed.
        Cell type I (blue-shaded, white border) obey reaction equations given by a
        substrate-depletion system~\cite{Schnakenberg1979} and are colored by their
        internal concentration of the activator.
        Cell type II (orange) is smaller than cell type I and is chemotactically
        attracted by the activator which is secreted by cell type I.
        The background displays the density profile of the secreted activator molecule
        (yellow: high density, blue: low density).
        The number of cells I is increased from subfigure A-D (256, 484, 1024, 2025),
        while the number of cells II remains fixed to 3000.
        Cell death reduces the overall number of agents.
        The pictures show the final state of the simulation after reaching (up to
        statistical fluctuations) a steady-state.
        The variations in cell number alone lead to different emerging patterns.
        While these results may be obtainable by a modified purely \ac{pde}-based
        approach, they are much easier to interpret and develop in an \ac{abm}.
        The simulations were carried out using PhysiCell~\cite{Ghaffarizadeh2018}.
    }
    \label{fig:abms-comparison-physicell-pde}
\end{figure}
%
% WHAT ARE CURRENT LIMITATIONS/CHALLENGES
\subsubsection*{Techniques and Challenges}
\label{subsec:analysis-challenges}\noindent
\acp{cabmf} allow researchers to investigate biological systems on the cellular level
with the option to implement many details, with the downside of substantial computational
cost. To combat this issue, all presented frameworks are of multi-scale nature.\\
The relevant time- and length-scales are identified and the corresponding sub-processes
are modeled and updated according to their scales.
This can greatly improve performance as for example diffusion-driven processes tend to be
much faster than cell migrational or phenotypical processes~\cite{Keener2002}.
Other techniques to improve performance are efficient $\mathcal{O}(N_{\mathrm{cells}})$
implementations of algorithms~\cite{Meagher1982} to calculate direct cell-cell
interaction partners~\cite{}, spreading the computational load over multiple processes
via multiprocessing (for example via OpenMP~\cite{Chandra2001}) or on specialized devices
such as solving \acp{pde} on a \ac{gpu}~\cite{Steuwer2011}.
%
% FEATURE EXTRACTION
Due to the stochastic nature of the \ac{abm} simulations, appropriate statistical methods
need to be applied, which is often challenged by the fact that transient developmental
processes are studied not necessarily reaching a stationary state.
Analysis of the simulation and comparison with experimental data requires the definition
of precise features which are extracted from the simulation results.
It is important to define clear goals and questions upfront, as this will guide the
process of feature extraction and dimensional reduction.
To this end machine learning techniques are becoming more and more popular for analysis
of \ac{abm} results~\cite{Efimenko2020,Caicedo2017}.
Given current advancements in machine learning, researchers are hopeful that image
classification of patterns and self-organizing systems can get more automated in the
future~\cite{Ker2018,Efimenko2020}.
The authors of TiSim/CellSys have explicitly suggested an image-to-model workflow~\cite{Hoehme2016}.
Neural networks showed promise in partly replacing analysis procedures~\cite{Chen2021}.
Other machine learning techniques can also be used to determine rules for agents and
calibrate the model~\cite{Sivakumar2022}.
Auto encoders~\cite{Kramer1991} may provide a way to obtain a dimensionally reduced
representation of complex \ac{abm} simulation results.
%
% SENSITITIVY ANALYSIS
Due to the mechanistic and 'law-driven' nature of \acp{abm}, often multiple unknown
parameters need to be determined or estimated from data.
Parameters can be estimated by comparing features extracted from experimental data and
from simulation results, which is already a substantial effort.
However, this process will usually yield uncertainties, which need to be quantified, as
it is not sufficient to evaluate the model locally in parameter space using a sometimes
arbitrarily chosen parameter set.
In order to focus on the relevant parameters, sensitivity analysis is an important tool,
which can also be used for model reduction~\cite{Saltelli2008}.
Due to highly integrated nature of \acp{abm} sensitivity analysis is demanding and
incorporates substantial computational costs~\cite{Aguilar2020}.
Consequently, it is often only possible to arrive at qualitative statements for complex
\ac{abm} simulations.

\section{Discussion}
\label{sec:discussion}\noindent
This review introduced the concepts of \aclp{abm} in cellular systems.
We compared different frameworks with respect to their conceptual and implementational differences.
To date, a large number of different \aclp{abmf} with different strengths and weaknesses
exist and are openly available.
The multitude of options is a clear indication for the overall interest in the subject.
\acp{abm} provide a unique tool to integrate combinations of processes and study their
respective dynamics.
Even for the exploration of systems that lack sufficient data, \acp{abm} can be used as
they can be developed initially with rather simplified rule sets, by means of which
researchers can generate hypotheses, which can in turn guide the design of laboratory experiments.
By this cycle of experimental and computational methods, the model and the experiments
can be improved and finally increase the conceptual knowledge about the system.
Due to this, it is important to understand the challenges of \acp{abm} and their limitations.
\acp{abm} can be seen as a mapping of specific rules to spatial configurations.
This mapping is non-unique, and the question arises, how the results of the ABM depend on
the set of rules and the used parameters.
How are the values (or distributions) of the parameters estimated? How does the
uncertainty in the system parameters affect the predictions of the simulations? In
particular, when analyzing the (often stochastic) results of a simulation, one needs to
quantify the influence of the parameter uncertainty which is a considerable challenge.
Besides these questions and challenges it can be expected that \acp{abm} are quickly
becoming mainstream tools in biology.
