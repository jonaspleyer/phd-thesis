\usepackage[utf8]{inputenc}

\usepackage{acronym}
\usepackage{amsmath}
\usepackage[USenglish]{babel}
\usepackage[style=ieee]{biblatex}
\usepackage{booktabs}
\usepackage{color}
\usepackage{float}
\usepackage{graphicx}
\usepackage[hidelinks]{hyperref}
\usepackage{listings}
\usepackage{listings-rust/listings-rust}
% No indent after paragraphs
\usepackage{parskip}
\usepackage{siunitx}

% /--- See https://www.tug.org/FontCatalogue/ ------------------------------------------------------
% \-------------------------------------------------------------------------------------------------

% /--- Everything about fancyhdr -------------------------------------------------------------------
\usepackage{fancyhdr}
% \fancyhead{} % clear all header fields
\fancyfoot{} % clear all footer fields
\fancyhead[LO,RE]{\rightmark}
\fancyhead[RO,LE]{Jonas Pleyer}
\fancyfoot[LO,RE]{\thepage}

\setlength{\headheight}{13.6pt}
% \-------------------------------------------------------------------------------------------------

% Configure lstlistings
\definecolor{beige}{RGB}{255, 245, 240}
\lstset{
    backgroundcolor=\color{beige},
    frame=single,
    keepspaces=true,
    captionpos=b,
    breaklines=true,
}

\addbibresource{references.bib}

\makeatletter
\title{Agent-Based Models in Cellular Systems}\let\Title\@title
\author{Jonas Pleyer}\let\Author\@author
\date{11.11.2025}\let\Date\@date
\makeatother

\begin{document}
% TITLEPAGE

\pagestyle{fancy}
\renewcommand{\sectionmark}[1]{\markright{\thesection~ ~#1}}
% \renewcommand{\chaptermark}[1]{\markboth{\chaptername~\thechapter~-~ #1}{}}

\begin{titlepage}
    % Turns off page numbering
    \pagenumbering{Alph}
    \thispagestyle{empty}
    \begin{center}

        \Large\textbf{ALBERT-LUDWIGS-UNIVERSITÄT\\ FREIBURG IM BREISGAU\\}
        \vspace{0.5cm}
        \Large\textbf{Institute of Physics}

        \rule{\textwidth}{1pt}
        \vspace{1.5cm}

        \huge\textbf{\Title}
        \Large\textbf{\subtitle}

        \vspace{1.5cm}

        \includegraphics[width=0.6\textwidth]{logos/uni-fr-sigil-black.png}

        %\vspace{18cm}
        \vfill

        \Large{Dissertation}\\
        \normalsize
        zur Erlangung des Doktorgrades\\
        % Doctoral Thesis in Physics\\
        \vspace{0.5cm}
        submitted \Date\hspace{0pt} by\\
        \vspace{0.5cm}
        \Large\textbf{Jonas Pleyer}\\
        \normalsize
        \vspace{0.5cm}
        % born in Heidelberg (Germany)\\
        \large Supervisor: Dr. Christian Fleck\\
        \large Supervisor: Prof. Dr. Jens Timmer\\
        \normalsize

        \newpage\null\thispagestyle{empty}
        % Here but anything like ISBN or whatever credentials of the final binding might be
        % required.
        \newpage
        \thispagestyle{empty}
        \newpage
        I thank my wife Anja for her unweary support and love.

    \end{center}
\end{titlepage}
\thispagestyle{empty}


%###################################################################################################
\begin{abstract}
    TODO
\end{abstract}

\newpage
\tableofcontents
\pagenumbering{Roman}
\newpage

\listoffigures
\listoftables
\lstlistoflistings

\section*{Acronyms}
\begin{acronym}
    \acro{abm}[ABM]{Agent-Based Model}
    \acro{ib}[IB]{Individual-Based}
    \acro{msd}[MSD]{Mean Squared Displacement}
\end{acronym}

\section*{Overview}

%###################################################################################################
\newpage
\pagenumbering{arabic}
\section{Introduction}

Kapitel 1 soll eine Einführung in ABMs geben. Dabei möchte ich vor Allem das Mini-Review
von uns verwenden und auf Gemeinsamkeiten, Unterschiede, und ein wenig Geschichte
eingehen. Hier möchte ich schon einige Ideen aufbringen, die später wieder aufgegriffen
werden. Diese habe ich erstmal als eigene Subsections gelistet.

%---------------------------------------------------------------------------------------------------
\subsection{Cellular Building Blocks}
\begin{itemize}
    \item complex organisms from collection of cells
    \item self-organization/pattern formation
    \item single-cell vs bulk
\end{itemize}

%---------------------------------------------------------------------------------------------------
\subsection{Comparison of Simulation Frameworks}
\begin{itemize}
    \item use mainly mini-review for this subsection \cite{Pleyer2023}
    \item many frameworks
    \item some similar approaches
    \item purpose-built solutions
    \item missing flexibility in model design or ability to apply to various systems
    \item large number of parameters which need to be supplied
\end{itemize}

%---------------------------------------------------------------------------------------------------
\subsection{Mathematical Treatment}
\begin{itemize}
    \item no "unfiying theory"
    \item with theoretical framework, we would be able to systematically discuss any of the following:
        \begin{itemize}
            \item coarse-graining
            \item uncertainty analysis
            \item dimensionality reduction
        \end{itemize}
\end{itemize}

%###################################################################################################
\pagebreak
\section{cellular\_raza: A novel Agent-Based Framework}
\cite{Pleyer2025}

“The purpose of abstracting is not to be vague, but to create a new semantic level in which one can be absolutely precise.”
― Edsger W. Dijkstra

Kapitel 2 gibt eine ausführliche Erklärung von cellular\_raza. Dazu werde ich das JOSS
Paper und die Dokumentation auf https://cellular-raza.com/ verwenden.
Ich werde hier auch auf Details wie die Implementierung und Performance eingehen.
Ein wichtiger Bestandteil wird auch die zugrundeliegende Denkweise, indem ich die zellulären systeme in ihre "Aspecte" aufteile.
Diese werde ich im Einleitungskapitel davor motiviert habe.

%---------------------------------------------------------------------------------------------------
\subsection{Goals/Justification}

%...................................................................................................
\subsubsection{Systems that noone can look at so far}

\textbf{Ideas for complex interactions}
\begin{itemize}
    \item complex interactions of various celltypes (such as funghus with plant cells i.e. Rust
        fungus)
    \item video about how rust fungus works \url{https://www.youtube.com/watch?v=AKY_pelBZek}
    \item pear rust \url{https://www.youtube.com/watch?v=f7MEO_4lHX8} (very nice images, include
        them!)
    \item bacteria on top of plant cells (find images for this)
    \item bacteria in gut (interaction with epithelial cells) (find images for this)
\end{itemize}

%...................................................................................................
\subsubsection{Simulation Aspects}
\begin{table}[H]
    \centering
    \begin{tabular}{lll}
        Aspect\\
        \toprule
        \textbf{(C) Cellular}\\
        \midrule
        Position    (Spatial representation of the cell)\\
        Velocity    (Velocity of the cell)\\
        Cell-Cycle\\
        Intracellular Reactions   &\\
        \midrule
        \textbf{(CC) Cell-Cell}\\
        Physical Interactions\\
        Contact Reactions\\
        \midrule
        \textbf{(CE) Cell-Environment}\\
        \midrule
        Coupling of Intra- and Extracellular Reactions\\
        External Force\\
        \textbf{(E) Environment}\\
        \midrule
        Fluid Dynamics \& Extracellular Reactions\\
        \textbf{(O) Other}\\
        \midrule
        Controller\\
        \bottomrule
    \end{tabular}
    \caption{TODO}
\end{table}

%---------------------------------------------------------------------------------------------------
\subsection{Structure \& Development}
\paragraph{Key concepts to keep in mind}
\begin{itemize}
    \item simulation aspects: cell, cell-cell, domain-cell, other
    \item flexibility
    \item automated testing, good documentation
    \item Rust as proramming language
\end{itemize}

\paragraph{Modular Layout} (Make this into a nice dependency graph)
\begin{enumerate}
    \item concepts
    \item building blocks
    \item core (solvers, etc.)
    \item main package, benchmarks
\end{enumerate}

%...................................................................................................
\subsubsection{Spatial Partitioning}
\begin{itemize}
    \item Why can we do that? $\rightarrow$ finite interactions
    \item How do we do this? $\rightarrow$ in-cell approach
    \item agnostic with respect to particular partitioning; use trait for this
    \item List examples of partitioning:
    \begin{itemize}
        \item Grid (this is what we do)
        \item k-d-tree
        \item quad tree
        \item STR \& STR+
        \item Z-curve
        \item Hilbert curve
    \end{itemize}
    \item discuss repartitioning (in the middle of one simulation)
    \item discuss parallelization wrt. partitioning (how do we bundle together various voxels for
        specific threads, minimize communication)
\end{itemize}

%...................................................................................................
\subsubsection{Algorithm \& Code Generation}
\begin{itemize}
    \item insert code as needed depending on which simulation aspects are solved
    \item macro-based approach
    \item modularize components
    \item allow for all kinds of agents and domains/environments
\end{itemize}

%...................................................................................................
\subsubsection{Rust}
\begin{itemize}
    \item modern language
    \item good ecosystem; allows for quick reusability and iterations
    \item safety, robustness
\end{itemize}

%---------------------------------------------------------------------------------------------------
\subsubsection{Functionality \& Limitations}

%---------------------------------------------------------------------------------------------------
\subsection{Benchmarks}
This section provides multiple benchmarks which can be read and unserstood separately.
Their purpose is to provide insights into performance characteristics and assert the correctness of
the solvers and routines used within \texttt{cellular\_raza}.

\subsubsection{Multithreading Performance (Amdahl's Law)}

\paragraph{Theory}
One measure of multithreaded performance is to calculate the possible theoretical speedup
given by Amdahl's law~\cite{Rodgers1985}.
It provides an estimate for the speedup and assumes that the workload can be split into a
parallelizable and non-parallelizable part which is quantified by $0\leq p \leq1$.
A higher value means that the contribution coming from non-parallelizable algorithms is lower.
The theoretical maximum $p=1$ means that all of the executed code is parallelizable.
Amdahl's law is given by

\begin{equation}
    T(n) = T_0\frac{1}{(1-p) + \frac{p}{n}}
    \label{eq:amdahls-law}
\end{equation}

where $T(n)$ describes the throughput which can be achieved given $n$ parallel threads and the
variable $p$ is the relative proportion of execution time which benefits from parallelization.
The total latency of a program can be determined via the inverse of the throughput.

\paragraph{Simulation Setup}
Measuring the performance of any simulation will be highly dependent on the specific cellular
properties and complexity.
For this comparison, we chose the previously explained cell-sorting example which contains minimal
complexity compared to other examples (see
\href{https://cellular-raza.com/showcase}{cellular-raza.com/showcase}).
Any computational overhead which is intrinsic to \texttt{cellular\_raza} and not related to the
chosen example would thus be more likely to manifest itself in performance results.

In order to produce reproducible results and simplify this overall process, we provide the
\texttt{cellular\_raza-benchmarks} crate.
It is a command-line utility which can be used to run benchmarks with various configurations.
Its arguments are displayed in Listing~\ref{listing:benchmarks-cli}.

\begin{minipage}{\linewidth}\begin{lstlisting}[
    language=Bash,
    basicstyle=\ttfamily\footnotesize,
    label=listing:benchmarks-cli,
    caption={[Usage of the benchmark CLI tool]
        Usage of the benchmark CLI tool.
        We provide two benchmarks, one for increasing the number of agents and another for
        increasing the number of threads.
        The subcommands can be further customized and will automatically run the given simulation
        multiple times for the specified configurations.
    }
]
Usage: cell_sorting [OPTIONS] <NAME> [COMMAND]

Commands:
  threads   Thread scaling benchmark
  sim-size  Simulation Size scaling benchmark
  help      Print this message or the help of the given subcommand(s)

Arguments:
  <NAME>  Name of the current runs such as name of the device to be benchmarked

Options:
  -o, --output-directory <OUTPUT_DIRECTORY>
          Output directory of benchmark results [default: benchmark_results]
  -s, --sample-size <SAMPLE_SIZE>
          Number of samples to be generated for each measurement [default: 5]
      --no-save
          Do not save results. This takes priority against the overwrite settings
      --overwrite
          Overwrite existing results
      --no-output
          Disables output
  -h, --help
          Print help
  -V, --version
          Print version
\end{lstlisting}\end{minipage}

Results generated in this way are stored inside the \texttt{benchmark\_results} folder.
In addition, we provide a python script \texttt{plotting/cell\_sorting.py} to quickly visualize
the obtained results.

\paragraph{Hardware}
We ran this benchmark on three distinct hardware configurations.
Although there exists a wide range of variables which could influence our measured runtime results,
we expect that the biggest effects are due to power-limits and variable frequency of the
central processing unit (CPU) (see Figure~\ref{tab:hardware-configurations}).
Both of these effects can be circumvented by choosing an artificially fixed frequency which is low
enough such that the total power limit of the CPU is never reached even when multiple cores are
under load.
While it is well known that other aspects such as cache-size and memory latency can have an impact
on absolute performance, they should however not introduce any significant deviations in terms of
relative performance scaling.

\begin{table}
    \centering
    \begin{tabular}{l c c c}
        CPU & Fixed Clockspeed & Memory Frequency & TDP\\
        \hline
        AMD Ryzen 3700X~\cite{AMDProductSpecifications} & $\SI{2200}{\mega\hertz}$
        & $\SI{3200}{\mega T\per\second}$ & $\SI{65}{\watt}$\\
        AMD Ryzen Threadripper 3960X~\cite{AMDProductSpecifications} & $\SI{2000}{\mega\hertz}$
        & $\SI{3200}{\mega T\per\second}$ & $\SI{280}{\watt}$\\
        Intel Core i7-12700H~\cite{Inteli712700H} & $\SI{2000}{\mega\hertz}$
        & $\SI{4800}{\mega T\per\second}$
        & $\SI{45}{\watt}$\\
    \end{tabular}
    \caption{List of tested hardware configurations.}
    \label{tab:hardware-configurations}
\end{table}

\paragraph{Results}

\begin{figure}[H]
    \centering
    \includegraphics[width=0.8\textwidth]{cellular_raza-homepage/static/benchmarks/thread_scaling.png}
    \caption[Throughput performance of the cell-sorting example]
    {Performance of the throughput $T(n)$ for increasing number of utilized threads $n$.}
    \label{fig:amdahls-law-fit}
\end{figure}

In figure~\ref{fig:amdahls-law-fit}, we fit Amdahl's law of equation~\ref{eq:amdahls-law} to our
measured datapoints and obtain the parameter $p$ from which the theoretical maximal speedup $S$ can
be calculated via

\begin{equation}
    S = \lim\limits_{n\rightarrow\infty} T(n) = \frac{1}{1-p}
    \label{eq:amdahls-law-maximum-speedup}
\end{equation}

The values for the maximum theoretical speedup are $S_\text{3700X}=13.64\pm1.73$,
$S_\text{3960X}=44.99\pm2.80$ and $S_\text{12700H}=34.74\pm5.05$.
Their uncertainty $\sigma(S)$ can be calculated via the standard gaussian propagation

\begin{equation}
    \sigma(S) = \frac{\sigma(p)}{(1-p)^2}
\end{equation}

where $\sigma(p)$ is the uncertainty of the parameter $p$ obtained via the fit in
figure~\ref{fig:amdahls-law-fit}.

\paragraph{Discussion}
In a real-world scenario, the perfect score of a fully parallelizable system with $p=1$ is
considered unobtainable due to effects such as the workload of the underlying operating system and
physical constraints.
Additionally, the findings of this study demonstrate that the calculated value of $p$ is contingent
on the specific hardware utilized, and a subset of the simulation code, amounting to $1-p$, is
inherently non-parallelizable.
This phenomenon can be partially attributed to the initial configuration of the simulation, which is
required to commence in a single-threaded manner.
Only subsequent to the generation of all respective
\href{https://cellular-raza.com/docs/cellular_raza_core/backend/chili/struct.SubDomainBox.html}
{subdomains} can the simulation transition to a state of execution involving multiple workers.
Furthermore, the termination of the simulation results in the freeing of resources which also
requires computational resources.
Additionally, this action locks the main routine until the completion of all individual threads.
In the context of \texttt{cellular\_raza}, all threads are currently utilizing a shared
barrier~\cite{GjengsetHurdles2018} to synchronize with each other.
This configuration results in a scenario in which a single worker has the potential to obstruct the
operations of all others.
This limitation may be addressed in future iterations of \texttt{cellular\_raza}.
It should be noted that this is merely an implementation detail and not a fundamental restriction.
Nonetheless, the aggregate acceleration $S$ obtained in our configurations allows for remarkably
effective parallelization.
This can be directly ascribed to the implementation and the fundamental assumption of
\texttt{cellular\_raza} that all interactions are strictly local, and subdomains only interact along
their borders.

\subsubsection{Scaling with Problem Size}
For interactions with infinite range the computational complexity~\cite{Knuth1976} of calculating
all interactions between every agent growths quadratic with the number of agents $\mathcal{O}(N^2)$.
This can be seen when writing down the most simplistic implementation of calculating interaction
forces.

\begin{minipage}{\linewidth}\begin{lstlisting}[
    language=Rust,
    basicstyle=\ttfamily\footnotesize,
    label=listing:benchmarks-cli,
    caption={
        [Naive calculation of interactions between agents.]
        A naive calculation of interactions between agents.
        The double for-loop iterates over all agents inside the simulation and thus results in
        $N(N-1)/2=(N^2-N)/2$ total iterations.
    }
]
for i in 0..n_agents {
    for j in 0..i {
        // Calculate interactions between agents
        let f1 = agents[i].calculate_interaction_between(agents[j]);
        let f2 = agents[j].calculate_interaction_between(agents[i]);

        // ...
    }
}
\end{lstlisting}\end{minipage}

\texttt{cellular\_raza} was designed bottom-up with the assumption that all interactions between
cellular agents are local.
This means we can assume that a single cell interacts only with others which are within a certain
proximity defined by the type of interaction.
This behaviour allows us to decompose the Domain into multiple voxels which contain cells.
The size of the voxels is determined by the length of the interaction between the cells.
This means, we only need to calculate interactions between every cell in one voxel and all
neighbouring voxels.
We assume that every voxel contains an average of $n_c$ cells per voxel with $N_v$ number
of voxels and $N_n$ of neighbouring voxels.
The total complexity is then $\mathcal{O}(N_vn_c^2N_n)$ where we can substitute the
total number of cells $N=N_vn_c$ to obtain a scaling of

\begin{equation}
    \mathcal{O}(NN_nn_c)
\end{equation}

The parameters $N_n,N_c$ can be assumed to be constant when fixing overall cell-density or at least
capped from above and thus, we expect a linear scaling $O(N)$ in the properly decomposed approach.
This graph illustrates how interactions between cells are calculated when assuming a
rectangular decomposition scheme.

\subsubsection{Accuracy Testing of Contact Reactions}
\subsubsection{Testing Stochastic Motion via the Fluctuation-Dissipation Theorem}
Stochastic motion on a particle level is often well described by Brownian3D or Langevin3D
dynamics~\cite{Brown1828,Lemons1997}.
The Fluctuation-Dissipation theorem~\cite{Callen1951} gives estimates the for \ac{msd} of a
collection of such particles and can thus be used to test the numerical
implementation.

\paragraph{Brownian Dynamics}
In the case of Brownian dynamics~\cite{Brown1828}
\begin{equation}
    \dot{X} = -\frac{D}{k_B T} \nabla V(X) + \sqrt{2D}R(t),
\end{equation}
Einstein predicted that the diffusion constant is proportional to the particles
mobility~\cite{Einstein1905}.
The \ac{msd} can be derived by applying the FDT to the probability density function (PDF) of a
brownian particle
\begin{equation}
    P(x,t) = \frac{1}{\sqrt{4\pi D t}}\exp\left(-\frac{(x-x_0)^2}{4Dt}\right).
\end{equation}
By calculating the fourier transform, we obtain the characteristic function
\begin{equation}
    G(k) = \int \text{e}^{ikx} P(x,t|x0)dx = \text{exp}(ikx_0 - k^2Dt).
\end{equation}
We obtain the moments by differentiating the characteristic function
$\kappa_n = (-i)^n\partial_k^n G(k)|_{k=0}$.
\begin{align}
    \kappa_1 &= x_0\\
    \kappa_2 &= 2Dt
\end{align}
For more than one spatial dimension, this approach can be generalized and we obtain
\begin{equation}
    \left<r^2(t)\right> = 2d D t
\end{equation}
where $d$ is the number of spatial dimensions of our system.

\paragraph{Langevin Dynamics}
The estimate of the \ac{msd} for Langevin~\cite{Lemons1997}
\begin{equation}
    m \ddot{X} = - \nabla V(X) - \lambda m \dot{X} + \sqrt{2m\lambda k_B T}R(t)
\end{equation}
is given by~\cite{VANKAMPEN2007}
\begin{equation}
    \left<(x(t)-x_0)^2\right> =
    v^2(0) \frac{\text{e}^{-\lambda t}}{\lambda^2}
    - \frac{d k_B T}{m\lambda^2}
    \left(1-\text{e}^{-\lambda t}\right)
    \left(3 - \text{e}^{-\lambda t}\right)
    + \frac{2 d k_B T}{m\lambda}t.
\end{equation}
It is now a matter of writing code such that the described system can be solved, analyzed and
tested.

\paragraph{Code}
\begin{itemize}
    \item Link to online documentation which contains the code
    \item Possibly include this in the Supplement?
\end{itemize}

\paragraph{Results}

\begin{figure}[H]
    \centering
    \includegraphics[width=0.49\textwidth]
    {cellular_raza-homepage/static/benchmarks/2024-08-testing-stochastic-motion/brownian_2d_3/mean-squared-displacement.png}
    \includegraphics[width=0.49\textwidth]
    {cellular_raza-homepage/static/benchmarks/2024-08-testing-stochastic-motion/langevin_2d_4/mean-squared-displacement.png}
    \includegraphics[width=0.49\textwidth]
    {cellular_raza-homepage/static/benchmarks/2024-08-testing-stochastic-motion/brownian_2d_3/trajectories.png}
    \includegraphics[width=0.49\textwidth]
    {cellular_raza-homepage/static/benchmarks/2024-08-testing-stochastic-motion/langevin_2d_4/trajectories.png}
    % \includegraphics[width=0.49\textwidth]
    %     {cellular_raza-homepage/static/benchmarks/2024-08-testing-stochastic-motion/brownian_2d_3/heatmap.png}
    % \includegraphics[width=0.49\textwidth]
    %     {cellular_raza-homepage/static/benchmarks/2024-08-testing-stochastic-motion/langevin_2d_4/heatmap.png}
    \caption[Stochastic motion simulation results]
    {
        Note that the displayed errorbars are not identical to the ones used for automated testing.
        We plot the standard error $\sigma / \sqrt{N}$ while the standard deviation $\sigma$ is used
        to compare numerical results.
    }
\end{figure}

The figures above show results for the brownian and langevin cases.
We first compare the MSD with its predicted values and obtain parameters by fitting the predictor
to our simulated values.
In the second plots, all trajectories of all particles are shown.
We can clearly see that the motions in the Brownian case are much more fluctuating while the curves
of the Langevin case are more smooth although this behaviour depends on the chosen parameters.
In the last step, a heatmap is shown which counts the number of times each bin is visited by a
particle.

\paragraph{Discussion}
In practice, our testing scheme is most sensitive in the early stages of simulation since the
deviation of the MSD is lowest at this point in time.
This is a practical challenge since the initial steps of the solver are less accurate and thus the
overall uncertainty of the numerically obtained results.

%###################################################################################################
\pagebreak
\section{ABM Theory}

\begin{itemize}
    \item theoretical/mathematical treatment of \acp{abm}
    \item characterize overarching concepts between frameworks (simulation concepts)
    \item formulate mathematical notions to capture many (not necessarily all) \acp{abm}
    \item discuss parameter estimation, coarse-graining and pattern formation along these ideas
    \item "Language of \acp{abm}"
\end{itemize}

%---------------------------------------------------------------------------------------------------
\subsection{Theoretical Framework}
\begin{itemize}
    \item use to-be-released paper
\end{itemize}

%---------------------------------------------------------------------------------------------------
\subsection{Parameter Estimation \& Sensitivity Analysis}
%---------------------------------------------------------------------------------------------------
\subsection{Coarse-Graining}
\begin{itemize}
    \item bacterial branching
\end{itemize}

%---------------------------------------------------------------------------------------------------
\subsection{Pattern Formation}

%###################################################################################################
\pagebreak
\section{Applications}

Kapitel 4 beinhaltet Anwendungsbeispiele. Mein Ziel hier ist, dass jedes dieser Beispiele
(subsections) mit dem Hintergrundwissen von Kapitel 1-3 lesbar ist. Dabei möchte ich hier
vor Allem zeigen, wie cellular\_raza verwendet wird und wie man die entwickelten
mathematischen Methoden auf die jeweiligen Beispiele anwenden kann. Inwiefern jedes der
Beispiele sowohl numerische Simulation als auch eine theoretische Abhandlung bekommt,
muss ich dann noch sehen.

%---------------------------------------------------------------------------------------------------
\subsection{Cell-Sorting}
%---------------------------------------------------------------------------------------------------
\subsection{Bacterial Rods}
\begin{itemize}
    \item Parameter Estimation
    \item Individual Treatment
    \item Apply this to Vortex formation~\cite{}
\end{itemize}
%---------------------------------------------------------------------------------------------------
\subsection{Bacterial Branching}
\begin{itemize}
    \item Discuss Coarse-Graining
    \item Pattern Formation
\end{itemize}

%---------------------------------------------------------------------------------------------------
\subsection{Puzzle Cells}

%###################################################################################################
\section{Discussion}

%###################################################################################################
\section{Conclusion}

% keywords can be removed
% \keywords{}

\newpage
\printbibliography

%###################################################################################################
\newpage

% \setcounter{section}{0}
\newcounter{suppsection}
\newcounter{supptable}
\newcounter{suppfigure}
\newcounter{suppequation}

% \addtocounter{suppsection}{\value{section}}
% \addtocounter{supptable}{\value{table}}
% \addtocounter{suppfigure}{\value{figure}}
% \addtocounter{suppequation}{\value{equation}}

\renewcommand{\thesection}{S\arabic{suppsection}}
\renewcommand{\thetable}{S\arabic{supptable}}%
\renewcommand{\thefigure}{S\arabic{suppfigure}}%
\renewcommand{\theequation}{S\arabic{suppequation}}%

\let\oldsection\section
\let\oldequation\equation
\let\oldtable\table
\let\oldfigure\figure
% \renewcommand{\section}{\@ifstar{\@oldsection}{\@stepcounter{suppsection}\@oldsection}}
\makeatletter
\renewcommand{\section}{\@ifstar{\oldsection*}{\stepcounter{suppsection}\oldsection}}
\renewcommand{\equation}{\@ifstar{\oldequation*}{\stepcounter{suppequation}\oldequation}}
\renewcommand{\table}{\@ifstar{\oldtable*}{\stepcounter{supptable}\oldtable}}
\renewcommand{\figure}{\@ifstar{\oldfigure*}{\stepcounter{suppfigure}\oldfigure}}
\makeatother

%###################################################################################################
\section{My Supplement}

\newpage
\thispagestyle{empty}
\thispagestyle{empty}

\section*{Declaration}
Herewith I affirm that the submitted thesis was written autonomously by
myself and that I did not use any other sources and auxiliaries than declared in
this work. Under the acknowledged rules of scientific work (lege artis) literal or
analogous content borrowed from the work of others was appropriately
identified. Furthermore, I insure that the submitted master thesis is not and was
never part of any other examination procedures, either complete or in substantial
parts.

\vspace{3cm}

\hfill
\begin{tabular}[t]{c}
    \rule{17em}{0.4pt}\\Place, Date
\end{tabular}
\hfill
\begin{tabular}[t]{c}
    \rule{17em}{0.4pt}\\Signature
\end{tabular}
\hfill


\end{document}
